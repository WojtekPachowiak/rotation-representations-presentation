%%%%%%%%%%%%%%%%%%%%%%%%%%%%%%%%%%%%%%%%

\newcommand{\aatuple}[4]{(\degree{#1},[#2,#3,#4])}

\newcommand{\aavec}[3]{[#1,#2,#3]}


\newcommand{\hl}[1]{\colorbox{gray!30!white}{\it{#1}}}

%%%%%%%%%%%%%%%%%%%%%%%%%%%%%%%%%%%%%%%%
% sets

\newcommand{\sthree}{\mathbb{S}^3}

\newcommand{\stwo}{\mathbb{S}^2}

\newcommand{\sone}{\mathbb{S}^1}

\newcommand{\sutwo}{\mathbb{SU}(2)}

\newcommand{\sothree}{\mathbb{SO}(3)}
\newcommand{\sotwo}{\mathbb{SO}(2)}

\newcommand{\R}{\mathbb{R}}

\newcommand{\Z}{\mathbb{Z}}

%%%%%%%%%%%%%%%%%%%%%%%%%%%%%%%%%%%%%%
% SHORTCUTS

\newcommand{\selfref}{thesis}

\newcommand{\CS}{CS}

\newcommand{\mX}{\mathbf{R}_X}
\newcommand{\mY}{\mathbf{R}_Y}
\newcommand{\mZ}{\mathbf{R}_Z}

%%%%%%%%%%%%%%%%%%%%%%%%%%%%%%%%%%%%%%

\newcommand{\overbar}[1]{\mkern 1.5mu\overline{\mkern-1.5mu#1\mkern-1.5mu}\mkern 1.5mu}


\newcommand{\trace}{\text{Tr}}

\newcommand{\degree}[1]{{#1}^{\circ}}

\newcommand{\rotaxisn}{\rotaxis{n}}

\newcommand{\rotaxis}[1]{\mathbf{#1}}

\newcommand{\lrotaxis}[1]{\mathbf{\lowercase{#1}}}
\newcommand{\grotaxis}[1]{\mathbf{\uppercase{#1}}}


\newcommand{\rotmat}[2]{\mathbf{R}(\rotaxis{#1},#2)}
\newcommand{\trotmat}[2]{\mathbf{R}^\top(\rotaxis{#1},#2) }


\newcommand{\rotmatII}[1]{
	\begin{bmatrix}
		\cos{#1} & -\sin{#1} \\
		\sin{#1} & \cos{#1}
	\end{bmatrix}
}

\newcommand{\complexnumb}[2]{
#1 + i#2
}

\newcommand{\vecsymb}[1]{{\mathbf{\lowercase{#1}}}}
\newcommand{\vv}[1]{{\mathbf{\lowercase{#1}}}}

\newcommand{\vecthree}[3]{
\begin{bmatrix}
	#1 &
	#2&
	#3
\end{bmatrix}}

\newcommand{\vecthreecol}[3]{
	\begin{bmatrix}
		#1 \\
		#2\\
		#3
\end{bmatrix}}

\newcommand{\vectwocol}[2]{
	\begin{bmatrix}
		#1 \\
		#2
\end{bmatrix}}

\newcommand{\tvectwocol}[2]{
	\begin{bmatrix}
		#1 &
		#2
\end{bmatrix}^\top}

\newcommand{\tvecthree}[3]{
\vecthree{#1}{#2}{#3}^\top	
}

\newcommand{\sphercoordsvec}{
	\tvecthree{\cos\alpha\sin\beta}{\sin\alpha\sin\beta}{\cos\beta}
}

\newcommand{\genmat}[1]{\mathbf{R}_{#1}}

\newcommand{\tgenmat}[1]{\mathbf{R}_{#1}^\top}


\newcommand{\pihalf}{\frac{\pi}{2}}


\newcommand{\norm}[1]{\left\lVert#1\right\rVert}

\newcommand{\interpolate}[1]{\twoheadrightarrow_{#1}}

\newcommand{\crossmat}[1]{\left[#1\right]_{\times}}

\newcommand{\crossmatfull}[3]{
\begin{bmatrix}
	0 & -#3 & #2 \\
	#3 & 0 & -#1 \\
	-#2 & #1 & 0 
\end{bmatrix}
}

%%%%%%%%%%%%%%%%%%%%%%%%%%%%%%%%%%%
% quaternions

\newcommand{\quattw}[2]{
	t_#2(#1)
}

\newcommand{\quatsw}[3]{
	s_#3(#1, #2)
}

\newcommand{\quat}[4]{
	\textstyle(#1,(#2,#3,#4))
}

\newcommand{\quatvec}[2]{
	\textstyle \left(#1, \mathbf{#2}\right)
}

\newcommand{\quatvecmanual}[2]{
	\textstyle(#1, #2)
}

\newcommand{\quatrotaxis}[2]{
\textstyle q(\rotaxis{#1},#2)
}


\newcommand{\quatrotaxisinv}[2]{
	\textstyle q^{-1}(\rotaxis{#1},#2)
}

\newcommand{\quataa}[2]{
\quatvecmanual{\cos\frac{#2}{2}}{\rotaxis{#1}\sin\frac{#2}{2}}
}

\newcommand{\idmat}{\mathbf{I}}

%%%%%%%%%%%%%%%%%%%%%%%%%%%%%%%%%%%%
% eulers 

\newcommand{\eulangovals}[3]{(\degree{#1},\degree{#2},\degree{#3})}
